
% Default to the notebook output style

    


% Inherit from the specified cell style.




    
\documentclass[11pt]{article}

    
    
    \usepackage[T1]{fontenc}
    % Nicer default font (+ math font) than Computer Modern for most use cases
    \usepackage{mathpazo}

    % Basic figure setup, for now with no caption control since it's done
    % automatically by Pandoc (which extracts ![](path) syntax from Markdown).
    \usepackage{graphicx}
    % We will generate all images so they have a width \maxwidth. This means
    % that they will get their normal width if they fit onto the page, but
    % are scaled down if they would overflow the margins.
    \makeatletter
    \def\maxwidth{\ifdim\Gin@nat@width>\linewidth\linewidth
    \else\Gin@nat@width\fi}
    \makeatother
    \let\Oldincludegraphics\includegraphics
    % Set max figure width to be 80% of text width, for now hardcoded.
    \renewcommand{\includegraphics}[1]{\Oldincludegraphics[width=.8\maxwidth]{#1}}
    % Ensure that by default, figures have no caption (until we provide a
    % proper Figure object with a Caption API and a way to capture that
    % in the conversion process - todo).
    \usepackage{caption}
    \DeclareCaptionLabelFormat{nolabel}{}
    \captionsetup{labelformat=nolabel}

    \usepackage{adjustbox} % Used to constrain images to a maximum size 
    \usepackage{xcolor} % Allow colors to be defined
    \usepackage{enumerate} % Needed for markdown enumerations to work
    \usepackage{geometry} % Used to adjust the document margins
    \usepackage{amsmath} % Equations
    \usepackage{amssymb} % Equations
    \usepackage{textcomp} % defines textquotesingle
    % Hack from http://tex.stackexchange.com/a/47451/13684:
    \AtBeginDocument{%
        \def\PYZsq{\textquotesingle}% Upright quotes in Pygmentized code
    }
    \usepackage{upquote} % Upright quotes for verbatim code
    \usepackage{eurosym} % defines \euro
    \usepackage[mathletters]{ucs} % Extended unicode (utf-8) support
    \usepackage[utf8x]{inputenc} % Allow utf-8 characters in the tex document
    \usepackage{fancyvrb} % verbatim replacement that allows latex
    \usepackage{grffile} % extends the file name processing of package graphics 
                         % to support a larger range 
    % The hyperref package gives us a pdf with properly built
    % internal navigation ('pdf bookmarks' for the table of contents,
    % internal cross-reference links, web links for URLs, etc.)
    \usepackage{hyperref}
    \usepackage{longtable} % longtable support required by pandoc >1.10
    \usepackage{booktabs}  % table support for pandoc > 1.12.2
    \usepackage[inline]{enumitem} % IRkernel/repr support (it uses the enumerate* environment)
    \usepackage[normalem]{ulem} % ulem is needed to support strikethroughs (\sout)
                                % normalem makes italics be italics, not underlines
    

    
    
    % Colors for the hyperref package
    \definecolor{urlcolor}{rgb}{0,.145,.698}
    \definecolor{linkcolor}{rgb}{.71,0.21,0.01}
    \definecolor{citecolor}{rgb}{.12,.54,.11}

    % ANSI colors
    \definecolor{ansi-black}{HTML}{3E424D}
    \definecolor{ansi-black-intense}{HTML}{282C36}
    \definecolor{ansi-red}{HTML}{E75C58}
    \definecolor{ansi-red-intense}{HTML}{B22B31}
    \definecolor{ansi-green}{HTML}{00A250}
    \definecolor{ansi-green-intense}{HTML}{007427}
    \definecolor{ansi-yellow}{HTML}{DDB62B}
    \definecolor{ansi-yellow-intense}{HTML}{B27D12}
    \definecolor{ansi-blue}{HTML}{208FFB}
    \definecolor{ansi-blue-intense}{HTML}{0065CA}
    \definecolor{ansi-magenta}{HTML}{D160C4}
    \definecolor{ansi-magenta-intense}{HTML}{A03196}
    \definecolor{ansi-cyan}{HTML}{60C6C8}
    \definecolor{ansi-cyan-intense}{HTML}{258F8F}
    \definecolor{ansi-white}{HTML}{C5C1B4}
    \definecolor{ansi-white-intense}{HTML}{A1A6B2}

    % commands and environments needed by pandoc snippets
    % extracted from the output of `pandoc -s`
    \providecommand{\tightlist}{%
      \setlength{\itemsep}{0pt}\setlength{\parskip}{0pt}}
    \DefineVerbatimEnvironment{Highlighting}{Verbatim}{commandchars=\\\{\}}
    % Add ',fontsize=\small' for more characters per line
    \newenvironment{Shaded}{}{}
    \newcommand{\KeywordTok}[1]{\textcolor[rgb]{0.00,0.44,0.13}{\textbf{{#1}}}}
    \newcommand{\DataTypeTok}[1]{\textcolor[rgb]{0.56,0.13,0.00}{{#1}}}
    \newcommand{\DecValTok}[1]{\textcolor[rgb]{0.25,0.63,0.44}{{#1}}}
    \newcommand{\BaseNTok}[1]{\textcolor[rgb]{0.25,0.63,0.44}{{#1}}}
    \newcommand{\FloatTok}[1]{\textcolor[rgb]{0.25,0.63,0.44}{{#1}}}
    \newcommand{\CharTok}[1]{\textcolor[rgb]{0.25,0.44,0.63}{{#1}}}
    \newcommand{\StringTok}[1]{\textcolor[rgb]{0.25,0.44,0.63}{{#1}}}
    \newcommand{\CommentTok}[1]{\textcolor[rgb]{0.38,0.63,0.69}{\textit{{#1}}}}
    \newcommand{\OtherTok}[1]{\textcolor[rgb]{0.00,0.44,0.13}{{#1}}}
    \newcommand{\AlertTok}[1]{\textcolor[rgb]{1.00,0.00,0.00}{\textbf{{#1}}}}
    \newcommand{\FunctionTok}[1]{\textcolor[rgb]{0.02,0.16,0.49}{{#1}}}
    \newcommand{\RegionMarkerTok}[1]{{#1}}
    \newcommand{\ErrorTok}[1]{\textcolor[rgb]{1.00,0.00,0.00}{\textbf{{#1}}}}
    \newcommand{\NormalTok}[1]{{#1}}
    
    % Additional commands for more recent versions of Pandoc
    \newcommand{\ConstantTok}[1]{\textcolor[rgb]{0.53,0.00,0.00}{{#1}}}
    \newcommand{\SpecialCharTok}[1]{\textcolor[rgb]{0.25,0.44,0.63}{{#1}}}
    \newcommand{\VerbatimStringTok}[1]{\textcolor[rgb]{0.25,0.44,0.63}{{#1}}}
    \newcommand{\SpecialStringTok}[1]{\textcolor[rgb]{0.73,0.40,0.53}{{#1}}}
    \newcommand{\ImportTok}[1]{{#1}}
    \newcommand{\DocumentationTok}[1]{\textcolor[rgb]{0.73,0.13,0.13}{\textit{{#1}}}}
    \newcommand{\AnnotationTok}[1]{\textcolor[rgb]{0.38,0.63,0.69}{\textbf{\textit{{#1}}}}}
    \newcommand{\CommentVarTok}[1]{\textcolor[rgb]{0.38,0.63,0.69}{\textbf{\textit{{#1}}}}}
    \newcommand{\VariableTok}[1]{\textcolor[rgb]{0.10,0.09,0.49}{{#1}}}
    \newcommand{\ControlFlowTok}[1]{\textcolor[rgb]{0.00,0.44,0.13}{\textbf{{#1}}}}
    \newcommand{\OperatorTok}[1]{\textcolor[rgb]{0.40,0.40,0.40}{{#1}}}
    \newcommand{\BuiltInTok}[1]{{#1}}
    \newcommand{\ExtensionTok}[1]{{#1}}
    \newcommand{\PreprocessorTok}[1]{\textcolor[rgb]{0.74,0.48,0.00}{{#1}}}
    \newcommand{\AttributeTok}[1]{\textcolor[rgb]{0.49,0.56,0.16}{{#1}}}
    \newcommand{\InformationTok}[1]{\textcolor[rgb]{0.38,0.63,0.69}{\textbf{\textit{{#1}}}}}
    \newcommand{\WarningTok}[1]{\textcolor[rgb]{0.38,0.63,0.69}{\textbf{\textit{{#1}}}}}
    
    
    % Define a nice break command that doesn't care if a line doesn't already
    % exist.
    \def\br{\hspace*{\fill} \\* }
    % Math Jax compatability definitions
    \def\gt{>}
    \def\lt{<}
    % Document parameters
    \title{project\_diamond\_prediction}
    
    
    

    % Pygments definitions
    
\makeatletter
\def\PY@reset{\let\PY@it=\relax \let\PY@bf=\relax%
    \let\PY@ul=\relax \let\PY@tc=\relax%
    \let\PY@bc=\relax \let\PY@ff=\relax}
\def\PY@tok#1{\csname PY@tok@#1\endcsname}
\def\PY@toks#1+{\ifx\relax#1\empty\else%
    \PY@tok{#1}\expandafter\PY@toks\fi}
\def\PY@do#1{\PY@bc{\PY@tc{\PY@ul{%
    \PY@it{\PY@bf{\PY@ff{#1}}}}}}}
\def\PY#1#2{\PY@reset\PY@toks#1+\relax+\PY@do{#2}}

\expandafter\def\csname PY@tok@w\endcsname{\def\PY@tc##1{\textcolor[rgb]{0.73,0.73,0.73}{##1}}}
\expandafter\def\csname PY@tok@c\endcsname{\let\PY@it=\textit\def\PY@tc##1{\textcolor[rgb]{0.25,0.50,0.50}{##1}}}
\expandafter\def\csname PY@tok@cp\endcsname{\def\PY@tc##1{\textcolor[rgb]{0.74,0.48,0.00}{##1}}}
\expandafter\def\csname PY@tok@k\endcsname{\let\PY@bf=\textbf\def\PY@tc##1{\textcolor[rgb]{0.00,0.50,0.00}{##1}}}
\expandafter\def\csname PY@tok@kp\endcsname{\def\PY@tc##1{\textcolor[rgb]{0.00,0.50,0.00}{##1}}}
\expandafter\def\csname PY@tok@kt\endcsname{\def\PY@tc##1{\textcolor[rgb]{0.69,0.00,0.25}{##1}}}
\expandafter\def\csname PY@tok@o\endcsname{\def\PY@tc##1{\textcolor[rgb]{0.40,0.40,0.40}{##1}}}
\expandafter\def\csname PY@tok@ow\endcsname{\let\PY@bf=\textbf\def\PY@tc##1{\textcolor[rgb]{0.67,0.13,1.00}{##1}}}
\expandafter\def\csname PY@tok@nb\endcsname{\def\PY@tc##1{\textcolor[rgb]{0.00,0.50,0.00}{##1}}}
\expandafter\def\csname PY@tok@nf\endcsname{\def\PY@tc##1{\textcolor[rgb]{0.00,0.00,1.00}{##1}}}
\expandafter\def\csname PY@tok@nc\endcsname{\let\PY@bf=\textbf\def\PY@tc##1{\textcolor[rgb]{0.00,0.00,1.00}{##1}}}
\expandafter\def\csname PY@tok@nn\endcsname{\let\PY@bf=\textbf\def\PY@tc##1{\textcolor[rgb]{0.00,0.00,1.00}{##1}}}
\expandafter\def\csname PY@tok@ne\endcsname{\let\PY@bf=\textbf\def\PY@tc##1{\textcolor[rgb]{0.82,0.25,0.23}{##1}}}
\expandafter\def\csname PY@tok@nv\endcsname{\def\PY@tc##1{\textcolor[rgb]{0.10,0.09,0.49}{##1}}}
\expandafter\def\csname PY@tok@no\endcsname{\def\PY@tc##1{\textcolor[rgb]{0.53,0.00,0.00}{##1}}}
\expandafter\def\csname PY@tok@nl\endcsname{\def\PY@tc##1{\textcolor[rgb]{0.63,0.63,0.00}{##1}}}
\expandafter\def\csname PY@tok@ni\endcsname{\let\PY@bf=\textbf\def\PY@tc##1{\textcolor[rgb]{0.60,0.60,0.60}{##1}}}
\expandafter\def\csname PY@tok@na\endcsname{\def\PY@tc##1{\textcolor[rgb]{0.49,0.56,0.16}{##1}}}
\expandafter\def\csname PY@tok@nt\endcsname{\let\PY@bf=\textbf\def\PY@tc##1{\textcolor[rgb]{0.00,0.50,0.00}{##1}}}
\expandafter\def\csname PY@tok@nd\endcsname{\def\PY@tc##1{\textcolor[rgb]{0.67,0.13,1.00}{##1}}}
\expandafter\def\csname PY@tok@s\endcsname{\def\PY@tc##1{\textcolor[rgb]{0.73,0.13,0.13}{##1}}}
\expandafter\def\csname PY@tok@sd\endcsname{\let\PY@it=\textit\def\PY@tc##1{\textcolor[rgb]{0.73,0.13,0.13}{##1}}}
\expandafter\def\csname PY@tok@si\endcsname{\let\PY@bf=\textbf\def\PY@tc##1{\textcolor[rgb]{0.73,0.40,0.53}{##1}}}
\expandafter\def\csname PY@tok@se\endcsname{\let\PY@bf=\textbf\def\PY@tc##1{\textcolor[rgb]{0.73,0.40,0.13}{##1}}}
\expandafter\def\csname PY@tok@sr\endcsname{\def\PY@tc##1{\textcolor[rgb]{0.73,0.40,0.53}{##1}}}
\expandafter\def\csname PY@tok@ss\endcsname{\def\PY@tc##1{\textcolor[rgb]{0.10,0.09,0.49}{##1}}}
\expandafter\def\csname PY@tok@sx\endcsname{\def\PY@tc##1{\textcolor[rgb]{0.00,0.50,0.00}{##1}}}
\expandafter\def\csname PY@tok@m\endcsname{\def\PY@tc##1{\textcolor[rgb]{0.40,0.40,0.40}{##1}}}
\expandafter\def\csname PY@tok@gh\endcsname{\let\PY@bf=\textbf\def\PY@tc##1{\textcolor[rgb]{0.00,0.00,0.50}{##1}}}
\expandafter\def\csname PY@tok@gu\endcsname{\let\PY@bf=\textbf\def\PY@tc##1{\textcolor[rgb]{0.50,0.00,0.50}{##1}}}
\expandafter\def\csname PY@tok@gd\endcsname{\def\PY@tc##1{\textcolor[rgb]{0.63,0.00,0.00}{##1}}}
\expandafter\def\csname PY@tok@gi\endcsname{\def\PY@tc##1{\textcolor[rgb]{0.00,0.63,0.00}{##1}}}
\expandafter\def\csname PY@tok@gr\endcsname{\def\PY@tc##1{\textcolor[rgb]{1.00,0.00,0.00}{##1}}}
\expandafter\def\csname PY@tok@ge\endcsname{\let\PY@it=\textit}
\expandafter\def\csname PY@tok@gs\endcsname{\let\PY@bf=\textbf}
\expandafter\def\csname PY@tok@gp\endcsname{\let\PY@bf=\textbf\def\PY@tc##1{\textcolor[rgb]{0.00,0.00,0.50}{##1}}}
\expandafter\def\csname PY@tok@go\endcsname{\def\PY@tc##1{\textcolor[rgb]{0.53,0.53,0.53}{##1}}}
\expandafter\def\csname PY@tok@gt\endcsname{\def\PY@tc##1{\textcolor[rgb]{0.00,0.27,0.87}{##1}}}
\expandafter\def\csname PY@tok@err\endcsname{\def\PY@bc##1{\setlength{\fboxsep}{0pt}\fcolorbox[rgb]{1.00,0.00,0.00}{1,1,1}{\strut ##1}}}
\expandafter\def\csname PY@tok@kc\endcsname{\let\PY@bf=\textbf\def\PY@tc##1{\textcolor[rgb]{0.00,0.50,0.00}{##1}}}
\expandafter\def\csname PY@tok@kd\endcsname{\let\PY@bf=\textbf\def\PY@tc##1{\textcolor[rgb]{0.00,0.50,0.00}{##1}}}
\expandafter\def\csname PY@tok@kn\endcsname{\let\PY@bf=\textbf\def\PY@tc##1{\textcolor[rgb]{0.00,0.50,0.00}{##1}}}
\expandafter\def\csname PY@tok@kr\endcsname{\let\PY@bf=\textbf\def\PY@tc##1{\textcolor[rgb]{0.00,0.50,0.00}{##1}}}
\expandafter\def\csname PY@tok@bp\endcsname{\def\PY@tc##1{\textcolor[rgb]{0.00,0.50,0.00}{##1}}}
\expandafter\def\csname PY@tok@fm\endcsname{\def\PY@tc##1{\textcolor[rgb]{0.00,0.00,1.00}{##1}}}
\expandafter\def\csname PY@tok@vc\endcsname{\def\PY@tc##1{\textcolor[rgb]{0.10,0.09,0.49}{##1}}}
\expandafter\def\csname PY@tok@vg\endcsname{\def\PY@tc##1{\textcolor[rgb]{0.10,0.09,0.49}{##1}}}
\expandafter\def\csname PY@tok@vi\endcsname{\def\PY@tc##1{\textcolor[rgb]{0.10,0.09,0.49}{##1}}}
\expandafter\def\csname PY@tok@vm\endcsname{\def\PY@tc##1{\textcolor[rgb]{0.10,0.09,0.49}{##1}}}
\expandafter\def\csname PY@tok@sa\endcsname{\def\PY@tc##1{\textcolor[rgb]{0.73,0.13,0.13}{##1}}}
\expandafter\def\csname PY@tok@sb\endcsname{\def\PY@tc##1{\textcolor[rgb]{0.73,0.13,0.13}{##1}}}
\expandafter\def\csname PY@tok@sc\endcsname{\def\PY@tc##1{\textcolor[rgb]{0.73,0.13,0.13}{##1}}}
\expandafter\def\csname PY@tok@dl\endcsname{\def\PY@tc##1{\textcolor[rgb]{0.73,0.13,0.13}{##1}}}
\expandafter\def\csname PY@tok@s2\endcsname{\def\PY@tc##1{\textcolor[rgb]{0.73,0.13,0.13}{##1}}}
\expandafter\def\csname PY@tok@sh\endcsname{\def\PY@tc##1{\textcolor[rgb]{0.73,0.13,0.13}{##1}}}
\expandafter\def\csname PY@tok@s1\endcsname{\def\PY@tc##1{\textcolor[rgb]{0.73,0.13,0.13}{##1}}}
\expandafter\def\csname PY@tok@mb\endcsname{\def\PY@tc##1{\textcolor[rgb]{0.40,0.40,0.40}{##1}}}
\expandafter\def\csname PY@tok@mf\endcsname{\def\PY@tc##1{\textcolor[rgb]{0.40,0.40,0.40}{##1}}}
\expandafter\def\csname PY@tok@mh\endcsname{\def\PY@tc##1{\textcolor[rgb]{0.40,0.40,0.40}{##1}}}
\expandafter\def\csname PY@tok@mi\endcsname{\def\PY@tc##1{\textcolor[rgb]{0.40,0.40,0.40}{##1}}}
\expandafter\def\csname PY@tok@il\endcsname{\def\PY@tc##1{\textcolor[rgb]{0.40,0.40,0.40}{##1}}}
\expandafter\def\csname PY@tok@mo\endcsname{\def\PY@tc##1{\textcolor[rgb]{0.40,0.40,0.40}{##1}}}
\expandafter\def\csname PY@tok@ch\endcsname{\let\PY@it=\textit\def\PY@tc##1{\textcolor[rgb]{0.25,0.50,0.50}{##1}}}
\expandafter\def\csname PY@tok@cm\endcsname{\let\PY@it=\textit\def\PY@tc##1{\textcolor[rgb]{0.25,0.50,0.50}{##1}}}
\expandafter\def\csname PY@tok@cpf\endcsname{\let\PY@it=\textit\def\PY@tc##1{\textcolor[rgb]{0.25,0.50,0.50}{##1}}}
\expandafter\def\csname PY@tok@c1\endcsname{\let\PY@it=\textit\def\PY@tc##1{\textcolor[rgb]{0.25,0.50,0.50}{##1}}}
\expandafter\def\csname PY@tok@cs\endcsname{\let\PY@it=\textit\def\PY@tc##1{\textcolor[rgb]{0.25,0.50,0.50}{##1}}}

\def\PYZbs{\char`\\}
\def\PYZus{\char`\_}
\def\PYZob{\char`\{}
\def\PYZcb{\char`\}}
\def\PYZca{\char`\^}
\def\PYZam{\char`\&}
\def\PYZlt{\char`\<}
\def\PYZgt{\char`\>}
\def\PYZsh{\char`\#}
\def\PYZpc{\char`\%}
\def\PYZdl{\char`\$}
\def\PYZhy{\char`\-}
\def\PYZsq{\char`\'}
\def\PYZdq{\char`\"}
\def\PYZti{\char`\~}
% for compatibility with earlier versions
\def\PYZat{@}
\def\PYZlb{[}
\def\PYZrb{]}
\makeatother


    % Exact colors from NB
    \definecolor{incolor}{rgb}{0.0, 0.0, 0.5}
    \definecolor{outcolor}{rgb}{0.545, 0.0, 0.0}



    
    % Prevent overflowing lines due to hard-to-break entities
    \sloppy 
    % Setup hyperref package
    \hypersetup{
      breaklinks=true,  % so long urls are correctly broken across lines
      colorlinks=true,
      urlcolor=urlcolor,
      linkcolor=linkcolor,
      citecolor=citecolor,
      }
    % Slightly bigger margins than the latex defaults
    
    \geometry{verbose,tmargin=1in,bmargin=1in,lmargin=1in,rmargin=1in}
    
    

    \begin{document}
    
    
    \maketitle
    
    

    
    \subsection{Udacity BAND \textbar{} Predicting diamond
prices}\label{udacity-band-predicting-diamond-prices}

    \subsubsection{Given model}\label{given-model}

    Price = -5,269 + 8,413 x Carat + 158.1 x Cut + 454 x Clarity

    Submit here:
https://classroom.udacity.com/nanodegrees/nd008/parts/235a5408-0604-4871-8433-a6d670e37bbf/project\#

    \subsubsection{Libraries}\label{libraries}

    \begin{Verbatim}[commandchars=\\\{\}]
{\color{incolor}In [{\color{incolor}1}]:} \PY{k+kn}{library}\PY{p}{(} ggplot2 \PY{p}{)}
\end{Verbatim}


    \subsection{Step 1: Understanding the
Model}\label{step-1-understanding-the-model}

\paragraph{According to the linear model provided, if a diamond is 1
carat heavier than another with the same cut and clarity, how much more
should we expect to pay?
Why?}\label{according-to-the-linear-model-provided-if-a-diamond-is-1-carat-heavier-than-another-with-the-same-cut-and-clarity-how-much-more-should-we-expect-to-pay-why}

    With the given model, an increase of 1 unit in carat while holding all
other covariates the same would increase the retail price of the diamond
by \$8413. The given linear model coefficient for carat indicates this
level of increase.

    \paragraph{If you were interested in a 1.5 carat diamond with a Very
Good cut (represented by a 3 in the model) and a VS2 clarity rating
(represented by a 5 in the model), how much would the model predict you
should pay for
it?}\label{if-you-were-interested-in-a-1.5-carat-diamond-with-a-very-good-cut-represented-by-a-3-in-the-model-and-a-vs2-clarity-rating-represented-by-a-5-in-the-model-how-much-would-the-model-predict-you-should-pay-for-it}

    \begin{Verbatim}[commandchars=\\\{\}]
{\color{incolor}In [{\color{incolor}2}]:} price \PY{o}{\PYZlt{}\PYZhy{}} \PY{l+m}{\PYZhy{}5269} \PY{o}{+} \PY{p}{(} \PY{l+m}{8413} \PY{o}{*} \PY{l+m}{1.5} \PY{p}{)} \PY{o}{+} \PY{p}{(} \PY{l+m}{158.1} \PY{o}{*} \PY{l+m}{3} \PY{p}{)} \PY{o}{+} \PY{p}{(} \PY{l+m}{454} \PY{o}{*} \PY{l+m}{5} \PY{p}{)}
        price
\end{Verbatim}


    10094.8

    
    \subsection{Step 2: Visualize the Data}\label{step-2-visualize-the-data}

    \paragraph{Plot 1 - Plot the data for the diamonds in the database, with
carat on the x-axis and price on the
y-axis.}\label{plot-1---plot-the-data-for-the-diamonds-in-the-database-with-carat-on-the-x-axis-and-price-on-the-y-axis.}

    \begin{Verbatim}[commandchars=\\\{\}]
{\color{incolor}In [{\color{incolor}3}]:} df \PY{o}{\PYZlt{}\PYZhy{}} read.csv\PY{p}{(}\PY{l+s}{\PYZsq{}}\PY{l+s}{../data/processed/diamonds.csv\PYZsq{}}\PY{p}{)}
        str\PY{p}{(} df \PY{p}{)}
\end{Verbatim}


    \begin{Verbatim}[commandchars=\\\{\}]
'data.frame':	50000 obs. of  8 variables:
 \$ X          : int  1 2 3 4 5 6 7 8 9 10 {\ldots}
 \$ carat      : num  0.51 2.25 0.7 0.47 0.3 0.33 2.01 0.51 1.7 0.53 {\ldots}
 \$ cut        : Factor w/ 5 levels "Fair","Good",..: 4 1 5 2 3 3 5 3 4 4 {\ldots}
 \$ cut\_ord    : int  4 1 3 2 5 5 3 5 4 4 {\ldots}
 \$ color      : Factor w/ 7 levels "D","E","F","G",..: 3 4 2 3 4 1 4 3 1 1 {\ldots}
 \$ clarity    : Factor w/ 8 levels "I1","IF","SI1",..: 5 1 6 5 7 3 3 8 3 6 {\ldots}
 \$ clarity\_ord: int  4 1 5 4 7 3 3 6 3 5 {\ldots}
 \$ price      : int  1749 7069 2757 1243 789 728 18398 2203 15100 1857 {\ldots}

    \end{Verbatim}

    \begin{Verbatim}[commandchars=\\\{\}]
{\color{incolor}In [{\color{incolor}4}]:} \PY{k+kp}{head}\PY{p}{(} df \PY{p}{)}
\end{Verbatim}


    \begin{tabular}{r|llllllll}
 X & carat & cut & cut\_ord & color & clarity & clarity\_ord & price\\
\hline
	 1         & 0.51      & Premium   & 4         & F         & VS1       & 4         & 1749     \\
	 2         & 2.25      & Fair      & 1         & G         & I1        & 1         & 7069     \\
	 3         & 0.70      & Very Good & 3         & E         & VS2       & 5         & 2757     \\
	 4         & 0.47      & Good      & 2         & F         & VS1       & 4         & 1243     \\
	 5         & 0.30      & Ideal     & 5         & G         & VVS1      & 7         &  789     \\
	 6         & 0.33      & Ideal     & 5         & D         & SI1       & 3         &  728     \\
\end{tabular}


    
    \begin{Verbatim}[commandchars=\\\{\}]
{\color{incolor}In [{\color{incolor}5}]:} ggplot\PY{p}{(} df\PY{p}{,} aes\PY{p}{(} x \PY{o}{=} carat\PY{p}{,} y \PY{o}{=} price \PY{p}{)} \PY{p}{)} \PY{o}{+}
            geom\PYZus{}point\PY{p}{(} alpha \PY{o}{=} \PY{l+m}{0.3} \PY{p}{)} \PY{o}{+}
            theme\PYZus{}bw\PY{p}{(}\PY{p}{)}
\end{Verbatim}


    
    
    \begin{center}
    \adjustimage{max size={0.9\linewidth}{0.9\paperheight}}{output_14_1.png}
    \end{center}
    { \hspace*{\fill} \\}
    
    \paragraph{Plot 2 - Plot the data for the diamonds for which you are
predicting prices with carat on the x-axis and predicted price on the
y-axis.}\label{plot-2---plot-the-data-for-the-diamonds-for-which-you-are-predicting-prices-with-carat-on-the-x-axis-and-predicted-price-on-the-y-axis.}

    \begin{Verbatim}[commandchars=\\\{\}]
{\color{incolor}In [{\color{incolor}6}]:} df.new \PY{o}{\PYZlt{}\PYZhy{}} read.csv\PY{p}{(} \PY{l+s}{\PYZsq{}}\PY{l+s}{../data/processed/new\PYZhy{}diamonds.csv\PYZsq{}} \PY{p}{)}
        str\PY{p}{(} df.new \PY{p}{)}
\end{Verbatim}


    \begin{Verbatim}[commandchars=\\\{\}]
'data.frame':	3000 obs. of  7 variables:
 \$ X          : int  1 2 3 4 5 6 7 8 9 10 {\ldots}
 \$ carat      : num  1.22 1.01 0.71 1.01 0.27 0.52 1.01 0.59 1.01 2.03 {\ldots}
 \$ cut        : Factor w/ 5 levels "Fair","Good",..: 4 2 5 3 3 4 4 3 2 3 {\ldots}
 \$ cut\_ord    : int  4 2 3 5 5 4 4 5 2 5 {\ldots}
 \$ color      : Factor w/ 7 levels "D","E","F","G",..: 4 4 6 1 5 4 3 1 2 3 {\ldots}
 \$ clarity    : Factor w/ 8 levels "I1","IF","SI1",..: 3 6 6 4 8 5 3 3 3 4 {\ldots}
 \$ clarity\_ord: int  3 5 5 2 6 4 3 3 3 2 {\ldots}

    \end{Verbatim}

    \begin{Verbatim}[commandchars=\\\{\}]
{\color{incolor}In [{\color{incolor}7}]:} \PY{k+kn}{attach}\PY{p}{(} df.new \PY{p}{)}
        df.new\PY{o}{\PYZdl{}}price \PY{o}{\PYZlt{}\PYZhy{}} \PY{l+m}{\PYZhy{}5269} \PY{o}{+} \PY{p}{(} \PY{l+m}{8413} \PY{o}{*} carat \PY{p}{)} \PY{o}{+} \PY{p}{(} \PY{l+m}{158.1} \PY{o}{*} cut\PYZus{}ord \PY{p}{)} \PY{o}{+} \PY{p}{(} \PY{l+m}{454} \PY{o}{*} clarity\PYZus{}ord \PY{p}{)}
        \PY{k+kp}{head}\PY{p}{(} df.new \PY{p}{)}
\end{Verbatim}


    \begin{tabular}{r|llllllll}
 X & carat & cut & cut\_ord & color & clarity & clarity\_ord & price\\
\hline
	 1         & 1.22      & Premium   & 4         & G         & SI1       & 3         & 6989.26  \\
	 2         & 1.01      & Good      & 2         & G         & VS2       & 5         & 5814.33  \\
	 3         & 0.71      & Very Good & 3         & I         & VS2       & 5         & 3448.53  \\
	 4         & 1.01      & Ideal     & 5         & D         & SI2       & 2         & 4926.63  \\
	 5         & 0.27      & Ideal     & 5         & H         & VVS2      & 6         &  517.01  \\
	 6         & 0.52      & Premium   & 4         & G         & VS1       & 4         & 1554.16  \\
\end{tabular}


    
    \begin{Verbatim}[commandchars=\\\{\}]
{\color{incolor}In [{\color{incolor}10}]:} ggplot\PY{p}{(}\PY{p}{)} \PY{o}{+}
             geom\PYZus{}point\PY{p}{(} data \PY{o}{=} df.new\PY{p}{,} aes\PY{p}{(} x \PY{o}{=} carat\PY{p}{,} y \PY{o}{=} price \PY{p}{)}\PY{p}{,} alpha \PY{o}{=} \PY{l+m}{0.2} \PY{p}{)} \PY{o}{+}
             theme\PYZus{}bw\PY{p}{(}\PY{p}{)}
\end{Verbatim}


    
    
    \begin{center}
    \adjustimage{max size={0.9\linewidth}{0.9\paperheight}}{output_18_1.png}
    \end{center}
    { \hspace*{\fill} \\}
    
    \subsubsection{Show both plots combined}\label{show-both-plots-combined}

    \begin{Verbatim}[commandchars=\\\{\}]
{\color{incolor}In [{\color{incolor}18}]:} df\PY{o}{\PYZdl{}}type \PY{o}{\PYZlt{}\PYZhy{}} \PY{l+s}{\PYZsq{}}\PY{l+s}{Original\PYZsq{}}
         df.new\PY{o}{\PYZdl{}}type \PY{o}{\PYZlt{}\PYZhy{}} \PY{l+s}{\PYZsq{}}\PY{l+s}{Predicted\PYZsq{}}
         
         df.c \PY{o}{\PYZlt{}\PYZhy{}} \PY{k+kp}{rbind}\PY{p}{(} df\PY{p}{,} df.new \PY{p}{)}
         df.c\PY{o}{\PYZdl{}}type \PY{o}{=} \PY{k+kp}{factor}\PY{p}{(} df.c\PY{o}{\PYZdl{}}type \PY{p}{)}
\end{Verbatim}


    \begin{Verbatim}[commandchars=\\\{\}]
{\color{incolor}In [{\color{incolor}19}]:} \PY{k+kp}{head}\PY{p}{(} df.c \PY{p}{)}
\end{Verbatim}


    \begin{tabular}{r|lllllllll}
 X & carat & cut & cut\_ord & color & clarity & clarity\_ord & price & type\\
\hline
	 1         & 0.51      & Premium   & 4         & F         & VS1       & 4         & 1749      & Original \\
	 2         & 2.25      & Fair      & 1         & G         & I1        & 1         & 7069      & Original \\
	 3         & 0.70      & Very Good & 3         & E         & VS2       & 5         & 2757      & Original \\
	 4         & 0.47      & Good      & 2         & F         & VS1       & 4         & 1243      & Original \\
	 5         & 0.30      & Ideal     & 5         & G         & VVS1      & 7         &  789      & Original \\
	 6         & 0.33      & Ideal     & 5         & D         & SI1       & 3         &  728      & Original \\
\end{tabular}


    
    \begin{Verbatim}[commandchars=\\\{\}]
{\color{incolor}In [{\color{incolor}20}]:} ggplot\PY{p}{(} df.c\PY{p}{,} aes\PY{p}{(} x \PY{o}{=} carat\PY{p}{,} y \PY{o}{=} price\PY{p}{,} colour \PY{o}{=} type \PY{p}{)} \PY{p}{)} \PY{o}{+}
                geom\PYZus{}point\PY{p}{(} alpha \PY{o}{=} \PY{l+m}{0.2} \PY{p}{)} \PY{o}{+}
                theme\PYZus{}bw\PY{p}{(}\PY{p}{)}
\end{Verbatim}


    
    
    \begin{center}
    \adjustimage{max size={0.9\linewidth}{0.9\paperheight}}{output_22_1.png}
    \end{center}
    { \hspace*{\fill} \\}
    
    \paragraph{What strikes you about this comparison? After seeing this
plot, do you feel confident in the model's ability to predict
prices?}\label{what-strikes-you-about-this-comparison-after-seeing-this-plot-do-you-feel-confident-in-the-models-ability-to-predict-prices}

    It strikes me that the model is not a great fit for the original data.
The model assumes linearity whereas the observed data seems to have some
curvature. I believe the model will overbid prices for diamonds around
.7 to 1 carat and underbid prices for diamonds above 1 carat. So no, I
don't feel confident in the models ability to accurately predict prices.
Other models would be better fits for these data.

    \subsection{Step 3: Make a
Recommendation}\label{step-3-make-a-recommendation}

    \paragraph{What price do you recommend the jewelry company to bid?
Please explain how you arrived at that number. HINT: The number should
be 7
digits.}\label{what-price-do-you-recommend-the-jewelry-company-to-bid-please-explain-how-you-arrived-at-that-number.-hint-the-number-should-be-7-digits.}

    \begin{Verbatim}[commandchars=\\\{\}]
{\color{incolor}In [{\color{incolor}22}]:} sum \PY{p}{(} df.new\PY{o}{\PYZdl{}}price \PY{p}{)} \PY{o}{*} \PY{l+m}{0.7}
\end{Verbatim}


    8213465.932

    
    Using the linear model provided, I would recommend the jewelry company
bidding \$8,213,465.93 for the diamonds. This number is the sum of the
predicted prices of the new diamonds times the wholesale discount at
70\%.


    % Add a bibliography block to the postdoc
    
    
    
    \end{document}
